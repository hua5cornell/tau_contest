\section{Introduction}
Timing-driven optimization is imperative for the success of closure flows, and it is essential to find efficient ways to make changes to the design to satisfy timing constraints and optimize power, performance and area. Buffer insertion and gate sizing are two of the most important and effective techniques in timing closure and design optimization and they have been well explored in many previous papers. 

According to the paper~\cite{Jiang:Interleaving}, buffer insertion implies an increase of the area, and might be more preferred for highly critical paths. For other mildly critical paths, however, gate sizing will be better choice. Thus, we are going to implement our design optimization flow in this contest as following: First, we run timing analysis for the whole circuit once, and identify high fan-out nets (and highly critical nets) with some reasonable criterions. Then we will perform buffer insertion for those high fan-out nets without an RC-tree, based on the contest rule that buffers cannot be inserted in an RC-tree. Finally, we will perform the gate sizing and conclude our design optimization flow.



%Also, according to the literature review, slew target method gives more efficient and accurate gate sizing solutions. Furthermore, our Galois system provides better support for graph-based algorithms. Considering all of these facts, we

