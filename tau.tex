\documentclass[conference]{IEEEtran}
\ifCLASSINFOpdf
\usepackage[pdftex]{graphicx}
\DeclareGraphicsExtensions{.eps}
\else
\usepackage[dvips]{graphicx}
\usepackage{multirow}
\DeclareGraphicsExtensions{.eps}
\fi
\usepackage{booktabs}
\RequirePackage[pdftex]{hyperref}
\usepackage[autostyle]{csquotes}
\usepackage{hyperref}
\usepackage{url}
\usepackage{tikz}
\usepackage{circuitikz}
\usepackage{caption}
\usepackage{diagbox}
\usetikzlibrary{calc, matrix, shapes, arrows, positioning, chains, decorations, shapes.gates.logic.US}
\newtheorem{algorithm}{Algorithm}
\newtheorem{mydef}{Definition}
\newtheorem{proposition}{Proposition}
\newtheorem{remark}{Remark}
\newtheorem{theorem}{Theorem}
\newtheorem{corollary}{Corollary}
\usepackage{algorithm}
\usepackage{algpseudocode}
\usepackage{amsmath}
\usepackage{amsfonts}
\usepackage{bm}
\usepackage{amssymb}
\usepackage{listings}
\usepackage{subcaption}
\lstset{
basicstyle={\normalsize\ttfamily},
escapeinside={(*}{*)},
%emph={function_up :}, emphstyle={\textbf}
}
\usepackage{ctable}

\begin{document}

\title{ParallelClosure: A Parallel Design Optimizer for Timing Closure\thanks{Thanks to various funding agencies. Omitted for blind review}
}

\author{\IEEEauthorblockN{Yi-Shan Lu}
\IEEEauthorblockA{\textit{Computer Science} \\
\textit{University of Texas at Austin}\\
Austin, TX \\
yishanlu@utexas.edu}
\and
\IEEEauthorblockN{Wenmian Hua}
\IEEEauthorblockA{\textit{Electrical Engineering} \\
\textit{Yale University}\\
New Haven, CT \\
wenmian.hua@yale.edu}
%\and
%\IEEEauthorblockN{3\textsuperscript{rd} Given Name Surname}
%\IEEEauthorblockA{\textit{dept. name of organization (of Aff.)} \\
%\textit{name of organization (of Aff.)}\\
%City, Country \\
%email address}
%\and
%\IEEEauthorblockN{4\textsuperscript{th} Given Name Surname}
%\IEEEauthorblockA{\textit{dept. name of organization (of Aff.)} \\
%\textit{name of organization (of Aff.)}\\
%City, Country \\
%email address}
}
\maketitle

\begin{abstract}
Timing-driven optimization is essential for the success of closure flows. In this contest, we explore existing efficient buffer insertion and gate sizing algorithms to fix timing violations, and minimize area, leakage power, and clock period for a design without altering its functionality. By using the Galois infrastructure for parallelization, we prototype a fast timing-driven design optimizer, ParallelClosure, to effectively and efficiently optimize designs with a reasonable number of timing violations.
\end{abstract}
\tikzset{
decision/.style={
  diamond,
  draw,
  text width=4em,
  text badly centered,
  inner sep=0pt
},
flipflop/.style={
  rectangle,
  draw,
  very thick,
  minimum height=5em,
  minimum width=1.3em,
},
block/.style={
  rectangle,
  draw,
  text width=6em,
  text centered,
  very thick,
  rounded corners,
  align=center,
},
cycle/.style={
  ellipse,
  draw,
  dashed,
  minimum width = 1.6cm,
  minimum height = 1.05cm,
  ultra thick,
},
Celement/.style={
  circle,
  draw,
  very thick,
  minimum width = 0.7cm,
},
combinational/.style={
  cloud,
  draw,
  very thick,
  minimum height=4.5em,
  minimum width=1.8cm,
},
module/.style={
  rectangle,
  draw,
  very thick,
  minimum height=5em,
  minimum width=1.2cm,
  inner sep=0pt,
},
cell/.style={
  rectangle,
  draw,
  very thick,
  minimum height=4.5em,
  minimum width=1cm,
  inner sep=0pt,
},
descr/.style={
  fill=white,
  inner sep=2.5pt,
},
CLK/.style={
  isosceles triangle,
  isosceles triangle apex angle=60,
  draw,
  very thick,
  minimum height=0.27cm,
  inner sep=0pt,
},
arrow/.style={
  ->, >=stealth, thick,
},
connector/.style={
  -latex,
  font=\scriptsize,
},
line/.style={
  draw, -latex',
  very thick,
  inner sep=0pt,
},
capa/.style={
  draw=black,
  line,
  minimum width=0.5cm,
  inner sep=0pt,
},
groud/.style={
  draw=black,
  line,
  minimum width=0.25cm,
  inner sep=0pt,
},
}
\section{Introduction}
Timing-driven optimization is imperative for the success of closure flows, and it is essential to find efficient ways to make changes to the design to satisfy timing constraints and optimize power, performance and area. In this contest, given a Verilog netlist, a SPEF file for net delay models, a set of cell libraries (corners) and an SDC file, we are going to generate an optimized design in Verilog and Spef with no violations of hold time and maximum slew transition, and minimized area, leakage power and clock period for all given corners. Note that the optimization should consider multi-corner multi-mode (MCMM) scenario.

The optimization can be done by buffer insertion/deletion and gate sizing. 


According to the paper~\cite{Jiang:Interleaving}, buffer insertion implies an increase of the area, and might be more preferred for highly critical paths. For other mildly critical paths, however, gate sizing will be better choice. Thus, we are going to implement our design optimization flow in this contest as following: First, we run timing analysis for the whole circuit once, and identify high fan-out nets (and highly critical nets) with some reasonable criterions. Then based on the contest rule that no buffers are allowed within an RC-tree, we will perform buffer insertion for those high fan-out nets without an RC-tree only and leave the remaining violations and optimization problems to gate sizing stage for convenience. Finally, we will perform the gate sizing and conclude our design optimization flow.



%Also, according to the literature review, slew target method gives more efficient and accurate gate sizing solutions. Furthermore, our Galois system provides better support for graph-based algorithms. Considering all of these facts, we


\section{Buffer Insertion}
\label{sec:buf_insert}
We use buffer insertion to first fix maximum capacitance violations, and then fix hold time violations.

To fix maximum capacitance violations, we run static timing analysis (STA) for the circuit and identify pins that are driving loads larger than their maximum capacitance limits. Then for each such pin, we insert a buffer in between the pin and the loads. The inserted buffer should be of the minimum size that can drive the load legally. If no such buffer exists, the buffer of the largest size is inserted to reduce the magnitude of violating the maximum capacitance limits.

To fix hold time violations, we adopt the technique presented in~\cite{Shenoy:Minimum}. We first perform STA and sort all the path endpoints by their worst hold time slacks across multiple corners. We then find the endpoint with worst negative hold time slack $s_h$, its corresponding path, and the associated corner. For all the edges on the path, we pick the edge $(i, j)$ that is a wire segment having the largest setup time slack ($s_s$) for the corresponding corner, and then insert buffers in between pins $i$ and $j$. To reduce the number of buffers inserted, we always insert the smallest buffer in the cell library. Since we are given cell libraries with non-linear delay model (NLDM), we approximate the delay of a buffer, $d_{buf}$, with the slew at $i$ and the pin capacitance at $j$. The number of buffers to be inserted is $\max\{1, \lceil\frac{\min\{|s_s|, |s_h|\}}{d_{buf}}\rceil\}$, which minimizes the impact to setup times. We repeat this process until all the hold time slacks are above a predefined threshold. In this contest, this threshold falls in the interval of $[30ps, 80ps]$ depending on the original worst hold time slack of the given design.



%In a circuit, for each pin $p$, $a_p$ denotes its early arrival time, $A_p$ denotes its late arrival time, $r_p$ denotes its early required time and $R_p$ denotes its late required time. For a legal circuit without timing violations, $r_p \leq a_p \leq A_p \leq R_p$. According to the paper~\cite{Shenoy:Minimum}, in any circuit, hold time violations can be fixed by buffer insertions if and only if $A_{i} - a_{i} \leq R_{o} - r_{o}$ where $i$ is a primary input and $o$ is a primary output or a register data input.


%In this contest, we adopt the technique presented in~\cite{Shenoy:Minimum} to insert buffers to remove hold time violations. In a circuit, for each pin $p$, $a_p$ denotes its early arrival time, $A_p$ denotes its late arrival time, $r_p$ denotes its early required time and $R_p$ denotes its late required time. For a legal circuit without timing violations, $r_p \leq a_p \leq A_p \leq R_p$. According to the paper~\cite{Shenoy:Minimum}, in any circuit, hold time violations can be fixed by buffer insertions if and only if $A_{i} - a_{i} \leq R_{o} - r_{o}$ where $i$ is a primary input and $o$ is a primary output or a register data input.

%\begin{theorem}
%In any circuit, let $i_1$ be a primary input and $i_n$ be a primary output such that there is a path $p$ from $i_1$ to $i_n$. The padding problem has a feasible solution if and only if for any such $p$, $A_{i1} - a_{i1} \leq R_{in} - r_{in}$.
%\end{theorem}

%According to the contest rule, buffers can be inserted an ideal wire, before the root of an RC-tree, or after the leaf of the RC-tree, but not within an RC-tree. In our implementation, we begin by an initial run of timing analysis for the circuit and identify pins that are driving loads larger than their maximum capacitance limit. Then for each such pin, we insert a buffer in between the pin and the loads. The inserted buffer should be of the minimum size that can drive the load legally.



% Since we can get the slew value of the edge from the timing analysis, and the capacitance value that the buffer is driving is also known, the two values are good approximations for the input slew and the output capacitance of the buffer, and we can use them to find the approximated delay value $d$ of the buffer. 


%\subsection{Buffer Insertion}
%%For buffer insertion problem, van Ginneken's algorithm~\cite{Lukas:Buffer} is considered to be a classic in the field. Given a fixed buffer tree and candidate buffer locations, it finds the optimal buffer solutions under Elmore delay models. Several papers extend the algorithm to handle library with multiple buffers, various kinds of constraints and delay models and wire sizing integration~\cite{Lillis:Optimal,Alpert:Wire,Alpert:Buffer,Lillis:Simultaneous,Cong:Buffer,Wong:A}. A significant limitation of van Ginneken's algorithm is that a fixed Steiner topology must be provided in advance. Several papers address the tree construction problems, including C-Tree Algorithm~\cite{Alpert:Buffered}, P-Tree Algorithm~\cite{Lillis:New} and S-Tree Algorithm~\cite{Hrkic:S}.
%
%In this contest, to minimize the area when inserting buffers, we only choose the smallest buffer available in the cell library. Also since we are considering the problem under multiple corners, we will choose the $\star$ worst corner to perform buffer insertion. Notice that this step is not aim to fix all the timing violations, but to just reduce the capacitance load of those high fan-out nets, and any remaining violations and optimization problems will be left to gate sizing section.
%
%We adopt the ideas presented in C-Tree Algorithm~\cite{Alpert:Buffered}, followed by van Ginneken's algorithm~\cite{Lukas:Buffer}. Since we are only going to insert buffers for high-fan-out nets without an RC-tree (which means ideal wires), we might want to modify the existing algorithm a little bit and we will not construct the specific steiner tree construction as described in~\cite{Alpert:Buffered}. We will start by performing 2-clustering, which separate the sinks with similar characteristics (criticality and capacitance) into two clusters recursively, from top to the bottom, until each cluster has no more than 2 sinks. This step essentially create a binary tree where each branching point is a potential position for a buffer. Then, the van Ginneken's algorithm~\cite{Lukas:Buffer} can be applied to determine the optimal buffer insertion solution for the given topology.
%
%Since the solution quality heavily depends on the initial topology we construct, we need to give a good clustering metric between pairs of points (sinks). Due to the ideal wire-model assumption of these nets for buffer insertion in this contest, our clustering metric is slightly different from that presented in~\cite{Lukas:Buffer}. In our implementation, sinks nodes are clustered based on the slack and the capacitance value. We adopt similar idea presented in the paper and the criticality distance between two sink nodes $s_i$ and $s_j$, $tDist(s_i, s_j) = |crit(s_i) - crit(s_j)|$ and
%\begin{equation}
%crit(s_i) = e^{\alpha(mAS - AS(s_i))/(aAS - mAS)}
%\end{equation}
%where $AS(s_i)$ is the slack of the sink node $s_i$, $mAS$ and $aAS$ are the minimum and the average slack for all the sinks, and $\alpha > 0$ is a user parameter. Since the buffers are inserted in an ideal wire in our problem, we also have $AS(s_i) = RAT(s_i)$, so are $mAS$ and $aAS$. Furthermore, our problem consider multiple corners, and we will just choose the corner which gives the worst slack for each sink so that the algorithm gives the buffer insertion solution under the worst case.
%
%The capacitance distance definition is different. Since we want to separate the two sinks whenever they lead to large total capacitance, a reasonable normalized capacitance distance definition between two sink nodes $s_i$ and $s_j$, $cDist(s_i, s_j)$ will be $|ncap(s_i) + ncap(s_j)|$, where
%\begin{equation}
%ncap(s_i) = \frac{cap(s_i)}{\Sigma_i cap(s_i)},
%\end{equation}
%The final distance $D(s_i, s_j)$ is simply defined as the summation of the two distances:
%\begin{equation}
%D(s_i, s_j) = tDist(s_i, s_j) + cDist(s_i, s_j).
%\end{equation}
%%we are going to use in clustering will be the Euclidean distance defined as:
%%\begin{equation}
%%D(s_i, s_j) = \sqrt{tDist(s_i, s_j)^2 + cDist(s_i, s_j)^2}.
%%\end{equation}








%\subsection{Buffer Insertion}
%%For buffer insertion problem, van Ginneken's algorithm~\cite{Lukas:Buffer} is considered to be a classic in the field. Given a fixed buffer tree and candidate buffer locations, it finds the optimal buffer solutions under Elmore delay models. Several papers extend the algorithm to handle library with multiple buffers, various kinds of constraints and delay models and wire sizing integration~\cite{Lillis:Optimal,Alpert:Wire,Alpert:Buffer,Lillis:Simultaneous,Cong:Buffer,Wong:A}. A significant limitation of van Ginneken's algorithm is that a fixed Steiner topology must be provided in advance. Several papers address the tree construction problems, including C-Tree Algorithm~\cite{Alpert:Buffered}, P-Tree Algorithm~\cite{Lillis:New} and S-Tree Algorithm~\cite{Hrkic:S}.
%
%In this contest, to minimize the area when inserting buffers, we only choose the smallest buffer available in the cell library. Also since we are considering the problem under multiple corners, we will choose the $\star$ worst corner to perform buffer insertion. Notice that this step is not aim to fix all the timing violations, but to just reduce the capacitance load of those high fan-out nets, and any remaining violations and optimization problems will be left to gate sizing section.
%
%We adopt the ideas presented in C-Tree Algorithm~\cite{Alpert:Buffered}, followed by van Ginneken's algorithm~\cite{Lukas:Buffer}. Since we are only going to insert buffers for high-fan-out nets without an RC-tree (which means ideal wires), we might want to modify the existing algorithm a little bit and we will not construct the specific steiner tree construction as described in~\cite{Alpert:Buffered}. We will start by performing 2-clustering, which separate the sinks with similar characteristics (criticality and capacitance) into two clusters recursively, from top to the bottom, until each cluster has no more than 2 sinks. This step essentially create a binary tree where each branching point is a potential position for a buffer. Then, the van Ginneken's algorithm~\cite{Lukas:Buffer} can be applied to determine the optimal buffer insertion solution for the given topology.
%
%Since the solution quality heavily depends on the initial topology we construct, we need to give a good clustering metric between pairs of points (sinks). Due to the ideal wire-model assumption of these nets for buffer insertion in this contest, our clustering metric is slightly different from that presented in~\cite{Lukas:Buffer}. In our implementation, sinks nodes are clustered based on the slack and the capacitance value. We adopt similar idea presented in the paper and the criticality distance between two sink nodes $s_i$ and $s_j$, $tDist(s_i, s_j) = |crit(s_i) - crit(s_j)|$ and
%\begin{equation}
%crit(s_i) = e^{\alpha(mAS - AS(s_i))/(aAS - mAS)}
%\end{equation}
%where $AS(s_i)$ is the slack of the sink node $s_i$, $mAS$ and $aAS$ are the minimum and the average slack for all the sinks, and $\alpha > 0$ is a user parameter. Since the buffers are inserted in an ideal wire in our problem, we also have $AS(s_i) = RAT(s_i)$, so are $mAS$ and $aAS$. Furthermore, our problem consider multiple corners, and we will just choose the corner which gives the worst slack for each sink so that the algorithm gives the buffer insertion solution under the worst case.
%
%The capacitance distance definition is different. Since we want to separate the two sinks whenever they lead to large total capacitance, a reasonable normalized capacitance distance definition between two sink nodes $s_i$ and $s_j$, $cDist(s_i, s_j)$ will be $|ncap(s_i) + ncap(s_j)|$, where
%\begin{equation}
%ncap(s_i) = \frac{cap(s_i)}{\Sigma_i cap(s_i)},
%\end{equation}
%The final distance $D(s_i, s_j)$ is simply defined as the summation of the two distances:
%\begin{equation}
%D(s_i, s_j) = tDist(s_i, s_j) + cDist(s_i, s_j).
%\end{equation}
%%we are going to use in clustering will be the Euclidean distance defined as:
%%\begin{equation}
%%D(s_i, s_j) = \sqrt{tDist(s_i, s_j)^2 + cDist(s_i, s_j)^2}.
%%\end{equation}

\section{Gate Sizing}
\label{sec:gate_sizing}

We use gate sizing to optimize for setup time, area and leakage power, while not introducing hold time violations and satisfying maximum slew requirement.
%Gate sizing problems are also well studied in the literature. TILOS~\cite{Fishburn:TILOS} applied Greedy method to gate sizing and modeled delay and slew functions as posynomials. The paper~\cite{Sapatnekar:An} provided a geometric programming approach, which is guaranteed to find the global optimal solution, to solve the transistor sizing problem. Lagrangian Relaxation is also widely used in both continuous gate sizing~\cite{Chen:Fast} and discrete gate sizing~\cite{Liu:A}, and it provides scalable methods to large instances. Faster approaches for gate sizing in practice given predefined discrete cell libraries are delay budget approaches~\cite{Chen:iCOACH,Dai:MOSIZ}, which repeatedly distribute delay budgets to cells and can manage interactions between gates~\cite{Held:Gate}. Slew target can also be distributed instead of delay budgets as an alternative, based on the assumption that the delay is a monotonic function of the slew. Most of other discrete gate sizing methods ignore the slew, and compared to these work, the slew targeting method enables a more accurate gate sizing~\cite{Held:Gate}.

In this contest, we implement the slew targeting method introduced by Held in~\cite{Held:Gate}, because (1) it directly addresses slew violations, (2) it avoids a large number of incremental timing updates, and (3) it is highly parallelizable. Since the paper focuses on fixing setup time violations, we generalize the method a little bit to handle hold time violations.

%Slew target approach~\cite{Held:Gate} also turns out to give more accurate gate sizing solutions than delay budget approach~\cite{Chen:iCOACH,Dai:MOSIZ}. Furthermore, it is a graph-based algorithm, which well adapts to our Galois system. Since the paper focuses on fixing setup time violations, we generalize the method a little bit such that it also addresses hold time violations.

\subsection{Rationale behind Slew Targeting}

Slew targeting works by associating each gate output pin with a slew target to guide the gate sizing process~\cite{Held:Gate}. To upsize/downsize a gate, just decrease/increase the slew target of its output pins. To facilitate MCMM optimization, each pin should have slew targets for all combinations of corners and delay modes.

The original slew targeting method assumes that delay and output slew are monotonic with respect to input slew. However, this assumption does not hold in general. For example, in the cell libraries given by the contest committee, this monotonicity does not hold when input slew is large and the output load is small for several cells. We think slew targeting still works in this case for two reasons: (1) The case when monotonicity does not hold is the situation we want to optimize away for the design; and (2) empirically, output slew and delay are weak functions of input slew, and the small fluctuations will not affect the accuracy of the final solution of our implementation by much.

In order for slew targeting to work, the following questions need to be addressed:
\begin{enumerate}
\item How to set the initial slew target for each gate output?
\item How to update the slew targets?
\item How to assign cells to gates?
\item When does the gate sizing converge?
\end{enumerate}

%Discrete gate sizing problem in the paper is stated as following: Given a circuit, we will assign each individual cell in the circuit a library cell from a discrete cell library, and the assignment should be chosen such that all the timing constraints are met (all the slack are non-negative) and the slew value at each pin is no larger than the slew limit value. Slew target method is based on the assumption that the delay is a monotonic function of the input slew. This assumption generally holds in practical design as can be seen in the cell library file, but there might be small fluctuations on delay/output slew vs input slew where the monotone property fail. Empirically, we do not have to worry about this because output slew and delay are weak functions of input slew, and the small fluctuations will not affect the accuracy of the final solution of our implementation by much.

%It turns out that slew target method gives more efficient and accurate gate sizing solutions. The assumption that the delay is a monotonic function of the slew generally holds, although some small fluctuations might exist, which should not affect the final solution by much. Furthermore, our Galois system provides better support for graph-based algorithms. Thus, in this contest, we are going to adopt the slew target method presented in~\cite{Held:Gate} for gate sizing.

\subsection{Slew Target Initialization}
We perform STA after buffer insertion and use the slew values as the initial slew targets.

\subsection{Slew Target Update}

Let $slewt(p)$ be the slew target of an output pin $p$ of gate $g$. In setup time analysis, if $p$ has a negative slack and it is on a critical path, we upsize $g$ by decreasing $slewt(p)$; otherwise, we downsize $g$ by increasing $slewt(p)$. In hold time analysis, the operation is opposite: downsize the gate when one of its output pins has negative slack and the pin is on a critical path, and upsize otherwise.

To know whether $p$ is on a critical path, we compute the local criticality of $p$, $lc(p)$, as follows. Let $slack^{-}(g) = \min\{slack(i) | (i, j) \in W \land j \in input(g)\}$, where $W$ is the set of wires and $input(g)$ is the set of input pins for gate $g$. Then $lc(p) = \max\{slack(p) - slack^{-}(g), 0\}$. Note that if $p$ belongs to a combinational gate $g$, then $lc(p) \geq 0$, and $p$ is on a critical path if and only if $lc(p) = 0$. Since we only optimize for datapaths in between registers, this definition of $lc(p)$ suffices to guide the update of slew targets.

Now we elaborate how to update the value for slew target of $p$. Instead of computing the change in slew target based on slacks as in~\cite{Held:Gate}, we leverage the information in the given cell libraries to directly come up with new slew targets.

Tables~\ref{table:lib1} and~\ref{table:lib2} show the rising slews for pin Z of cell BUF\_X1 and that of BUF\_X2, respectively. Note that (1) both tables contain similar values, and (2) for a given input slew and output capacitance, the corresponding entry in Table~\ref{table:lib2} is left to the one in Table~\ref{table:lib1}, and the two entries are always on the same row. For example, for input slew value of 15.6743 and output capacitance value of 30.3269, the output slew is given by the entry at row 3 and column 6 in Table~\ref{table:lib1}, and by the entry at row 3 and column 5 in Table~\ref{table:lib2}. This relation also holds for other types of functionally equivalent cells with different sizes across all corners. Therefore, the slew target of $p$ can be updated according to a sequence of numbers derived through table look-up into output slew tables using current input slew and different output capacitance. We call this sequence the slew possibilities of $p$.

To upsize/downsize a gate $g$ to which pin $p$ belongs, we take smaller/larger terms in $p$'s slew possibilities, which can be approximated as follows. Through table look-up, the smallest term, $lb$, is computed using the current input slew and zero driving capacitance; and the largest term, $ub$, using the current input slew and the maximum driving capacitance of $p$. Intermediate terms can be approximated based on a geometric sequence whose common ratio $r = (\frac{ub}{lb})^{\frac{1}{k}}$, where $k$ is a tunable parameter. In this contest, we have 8 terms in slew possibilities, and we set $k = 20$. The intermediate 6 terms are $lb\cdot r$, $lb\cdot r^3$, $lb\cdot r^5$, $lb\cdot r^8$, $lb\cdot r^{11}$, and $lb\cdot r^{15}$.

Recall that each output pin may have different slew targets for different timing corners and delay modes. These slew targets can be set independently from each other.

By updating the slew targets properly, we can eliminate maximum slew violation by construction.

\begin{table*}
\caption{The Rising Slew Table for BUF\_X1 from Nangate 45nm Typical Corner}
\label{table:lib1}
\centering
\begin{tabular}{|c|c|c|c|c|c|c|c|} \hline
\diagbox{Input Transition}{Output Capacitance} & 0.365616 & 1.895430 & 3.790860 & 7.581710 & 15.163400 & 30.326900 & 60.653700 \\ \hline
1.23599 & 3.33809 & 5.59725 & 8.60523 & 14.8575 & 27.5164 & 52.8765 & 103.604 \\ \hline
4.43724 & 3.33727 & 5.59699 & 8.60578 & 14.8576 & 27.5188 & 52.8775 & 103.599 \\ \hline
15.6743 & 3.40246 & 5.62543 & 8.61689 & 14.8582 & 27.517 & 52.8787 & 103.599 \\ \hline
37.1331 & 4.36023 & 6.10464 & 8.84317 & 14.9465 & 27.5247 & 52.8726 & 103.605 \\ \hline
70.5649 & 5.85455 & 7.27833 & 9.43026 & 15.0988 & 27.6409 & 52.9322 & 103.603 \\ \hline
117.474 & 7.61897 & 9.14083 & 10.8314 & 15.5462 & 27.6912 & 53.0238 & 103.669 \\ \hline
179.199 & 9.58764 & 11.3565 & 13.0249 & 16.7347 & 27.8716 & 53.0513 & 103.775 \\ \hline
\end{tabular}
\vspace{-1em}
\end{table*}

\begin{table*}
\caption{The Rising Slew Table for BUF\_X2 from Nangate 45nm Typical Corner}
\label{table:lib2}
\centering
\begin{tabular}{|c|c|c|c|c|c|c|c|} \hline
\diagbox{Input Transition}{Output Capacitance} & 0.365616 & 3.786090 & 7.572190 & 15.144400 & 30.288800 & 60.577500 & 121.155000 \\ \hline
1.23599 & 3.10917 & 5.67693 & 8.71288 & 14.9785 & 27.635 & 52.969 & 103.657 \\ \hline
4.43724 & 3.10875 & 5.67786 & 8.71402 & 14.9788 & 27.6339 & 52.9719 & 103.66 \\ \hline
15.6743 & 3.20354 & 5.70984 & 8.72471 & 14.9811 & 27.631 & 52.9744 & 103.651 \\ \hline
37.1331 & 4.20264 & 6.15463 & 8.94062 & 15.0761 & 27.6468 & 52.967 & 103.666 \\ \hline
70.5649 & 5.70174 & 7.27713 & 9.47332 & 15.2076 & 27.7634 & 53.0379 & 103.659 \\ \hline
117.474 & 7.47026 & 9.1372 & 10.8172 & 15.6132 & 27.8134 & 53.1232 & 103.735 \\ \hline
179.199 & 9.44195 & 11.3787 & 12.9969 & 16.7387 & 27.9813 & 53.162 & 103.831 \\ \hline
\end{tabular}
\vspace{-1em}
\end{table*}

%Specifically, look at Figure~\ref{fig:slew}. When sizing the cell $c$, the downstream cells are already sized and the output capacitance is already available. However, the input slew $slew(q)$ also depends on the unknown predecessor size. As a result, the paper gives an estimation of slew at $p'$, under setup time constraint (and hold time constraint): $est\_slew_s(p')$ (and $est\_slew_h(p')$), based on the slew provided by timing analysis and its slew target, to decide the size of $c$:
%\begin{equation}
%est\_slew_s(p') = \theta slewt_s(p') - (1-\theta)slew(p'),
%\end{equation}
%and
%\begin{equation}
%est\_slew_h(p') = \theta slewt_h(p') - (1-\theta)slew(p'),
%\end{equation}
%where $\theta \in [0, 1]$ is a factor which can shift from pure slew target $\theta = 1$ to real timing analysis $\theta = 0$. Empirically, $\theta$ should be close to 1 initially since the cell sizes are actively changing, but close to 0 later when the cell sizes converge.

\subsection{Cell Assignment}

\subsubsection{Order of Sizing}
\label{sec:sizing_order}

Having the slew target updated, we now size the gates. Since output capacitance usually has larger impact on the cell timing than the input slew, we want to fix all the downstream gates of $g$ before sizing $g$. This requires a reverse topological order of all gates. However, if all the combinational logic and registers are included, then the circuit topology has cycles. We cut the cycles at the data inputs to registers in order to work with a directed acyclic graph (DAG), from which we define a reverse topological order for gates.

\subsubsection{Input Slew Estimation}
As the given cell libraries are of NLDM, and upstream gates of $g$ are not fixed yet, we need to estimate slews for each of $g$'s input pins. Figure~\ref{fig:slew} shows an example. Let $est\_slew(p) = \theta slewt(p) + (1-\theta)slew(p)$ be the estimated slew at output pin $p$, where $slew(p)$ is the slew of $p$ from STA, and $\theta \in [0,1]$ is a parameter that is close to 1 when gate sizing starts but close to 0 when gate sizing is about to converge. Then, $est\_slew(q) = est\_slew(p') + slew\_degrad(p', q)$, where $slew\_degrad(p', q)$ denotes the slew degradation due to the wire segment from $p'$ to $q$, which can be computed by the RC-tree model~\cite{TAU:Contest}.

\subsubsection{Cell Selection}

Let $out\_est\_slew(p)$ be the estimated output slew at pin $p$ through table look-up using the load seen by $p$ and $est\_slew(q)$, where $(q, p)$ is a timing arc. A pin may have different slew targets for each combination of timing corners and setup/hold time analysis, so use the corresponding one when sizing a gate at a specific corner.

\begin{itemize}
\item In setup time analysis, we select for gate $g$ the minimum size of equivalent cells such that $out\_est\_slew(p) \leq slewt(p)$ holds for all output pins $p$ of gate $g$ at the corner being considered. Then we choose the maximum sized cell from the selected cells across all corners for gate $g$ in setup time analysis. We define $size_{s}(g)$ as the cell size selected for $g$ in setup time analysis.

\item In hold time analysis, the selection works in the opposite way: we select the maximum size of equivalent cells such that $out\_est\_slew(p) \geq slewt(p)$ holds for all output pins $p$ of gate $g$ at the corner being considered; and then choose the minimum sized cell from the selected cells across all corners for gate $g$ in hold time analysis. We define $size_{h}(g)$ as the cell size selected for $g$ in hold time analysis.

\item If $size_{s}(g) \leq size_{h}(g)$, then assign the cell of size $size_{s}(g)$ to gate $g$. This is because any cell sizes in $[size_{s}(g), size_{h}(g)]$ satisfy both slew targets in setup time and hold time analysis. In order to minimize area and leakage power, we choose $size_{s}(g)$ for gate $g$.

\item If $size_{s}(g) > size_{h}(g)$, then assign the cell of size $size_{h}(g)$ to gate $g$. This is because there is no cell size that can satisfy both slew targets in setup time and hold time analysis at the same time. We honor hold time constraints in this case without sacrificing setup time seriously, since hold time violations cannot be fixed by adjusting the clock frequency.

\item The cell size for gate $g$ should not violate maximum capacitance limits for any of $g$'s output pins. Otherwise, the cell of minimum size without such violation is chosen for gate $g$.
\end{itemize}

%\begin{table*}
%\caption{The Delay Table for Buf\_X1 from Nangate 45nm Typical Corner}
%\label{table:lib}
%\centering
%\begin{tabular}{|c|c|c|c|c|c|c|c|} \hline
%\diagbox{Input Transition}{Output Capacitance} & 0.365616 & 1.895430 & 3.790860 & 7.581710 & 15.163400 & 30.326900 & 60.653700 \\ \hline
%1.23599 & 3.33809 & 5.59725 & 8.60523 & 14.8575 & 27.5164 & 52.8765 & 103.604 \\ \hline
%4.43724 & 3.33727 & 5.59699 & 8.60578 & 14.8576 & 27.5188 & 52.8775 & 103.599 \\ \hline
%15.6743 & 3.40246 & 5.62543 & 8.61689 & 14.8582 & 27.517 & 52.8787 & 103.599 \\ \hline
%37.1331 & 4.36023 & 6.10464 & 8.84317 & 14.9465 & 27.5247 & 52.8726 & 103.605 \\ \hline
%70.5649 & 5.85455 & 7.27833 & 9.43026 & 15.0988 & 27.6409 & 52.9322 & 103.603 \\ \hline
%117.474 & 7.61897 & 9.14083 & 10.8314 & 15.5462 & 27.6912 & 53.0238 & 103.669 \\ \hline
%179.199 & 9.58764 & 11.3565 & 13.0249 & 16.7347 & 27.8716 & 53.0513 & 103.775 \\ \hline
%\end{tabular}
%\vspace{-1em}
%\end{table*}

%\begin{table*}
%\caption{The Delay Table for Buf\_X2 from Nangate 45nm Typical Corner}
%\label{table:lib}
%\centering
%\begin{tabular}{|c|c|c|c|c|c|c|c|} \hline
%\diagbox{Input Transition}{Output Capacitance} & 0.365616 & 3.786090 & 7.572190 & 15.144400 & 30.288800 & 60.577500 & 121.155000 \\ \hline
%1.23599 & 3.10917 & 5.67693 & 8.71288 & 14.9785 & 27.635 & 52.969 & 103.657 \\ \hline
%4.43724 & 3.10875 & 5.67786 & 8.71402 & 14.9788 & 27.6339 & 52.9719 & 103.66 \\ \hline
%15.6743 & 3.20354 & 5.70984 & 8.72471 & 14.9811 & 27.631 & 52.9744 & 103.651 \\ \hline
%37.1331 & 4.20264 & 6.15463 & 8.94062 & 15.0761 & 27.6468 & 52.967 & 103.666 \\ \hline
%70.5649 & 5.70174 & 7.27713 & 9.47332 & 15.2076 & 27.7634 & 53.0379 & 103.659 \\ \hline
%117.474 & 7.47026 & 9.1372 & 10.8172 & 15.6132 & 27.8134 & 53.1232 & 103.735 \\ \hline
%179.199 & 9.44195 & 11.3787 & 12.9969 & 16.7387 & 27.9813 & 53.162 & 103.831 \\ \hline
%\end{tabular}
%\vspace{-1em}
%\end{table*}

\subsection{Convergence}
\label{sec:sizing_converge}
After cell selection for all gates, we run STA again. If the worst negative slack improves for all corners, we proceed to next round of gate sizing using the current cell assignment. Otherwise, we score the current cell assignment by a linear combination of worst negative slack, average total negative slack over all path endpoints, and average cell area over all gates. The lower the score, the better. If the score decreases for all corners, we also proceed to next round of gate sizing using the current cell assignment. If not, we revert to the previous cell assignment and quit gate sizing.

%Again, since our problem considers multiple corners, after we get the sizing solution for each corner, we must take the maximum sizing for each cell across different corners to make sure that the sizing solution satisfies the slew targets for all the corners.

%Since the paper focuses on fixing setup time violations. We generalize the method by using similar method such that it also address the hold time violation. Instead, $slewt(p)$ are initialized to 0 and the cell assignment is also opposite to that in setup violation removal step: we choose the maximum cell size that gives the output slew $slew(p)$ such that $slew(p) \geq slewt(p)$, and take the minimum sizing for each cell across different corners to satisfy the slew target for hold time constraint for all the corners.
%
%The gate sizing solution under setup time constraint gives the upper bound for sizing of each cell, and that under hold time constraint gives the lower bound for sizing of each cell. I
%Initially, when the sizing of predecessors are actively changing, both $slewt(p')$ and $slew(p')$ contribute much to the actual slew at $p'$. Later when the sizing of predecessors become stable,


%The slew targets $slewt(p')$ and $slewt(p)$ are assigned for both $p'$ and $p$ and we can then use the formula presented in~\cite{TAU:Contest} to calculate $q$. To assign a size to the cell $c$, notice that at this time, the downstream cells have all been assigned and the output capacitance seen at $p$ is fixed and known. We can then use the table look-up in the cell library to determine the smallest cell size that gives the output slew $slew(p)$ such that $slew(p) \leq slewt(p)$. After gate(cell) sizing assignment, the timing analysis based on the design with new gate(cell) sizes, and refine the slew target for the next round of gate sizing assignment. This iterative process terminates until some stopping criterion is met.
\begin{figure}
\center
\begin{tikzpicture}[align=center, font=\footnotesize, node distance = 2cm]
  \node [module] (1) {}    ;
  \node [module] (2) [below right=-0.9cm and 1.5cm of 1] {$c$}  ;
  \draw [very thick] (1.east) -- ++(1, 0) node[pos=0.0, above right]{$p'$} -- ++(0, 0.4) -- ++(0.5, 0);
  \draw [very thick] (1.east) -- ++(1, 0) -- ++(0, -0.4) -- ++(0.5, 0) node[pos=0.1, below right]{$q$};
  \draw [very thick] ([yshift=-0.4cm]2.west) -- ++(-0.5, 0);
  \draw [very thick] ([yshift=-0.4cm]1.west) -- ++(-0.5, 0);
  \draw [very thick] ([yshift=0.4cm]1.west) -- ++(-0.5, 0);
  \draw [very thick] (2.east) -- ++(0.5, 0) node[pos=0.0, above right]{$p$};
\end{tikzpicture}
\caption{A Cell $c$ and a Predecessor}
\label{fig:slew}
\end{figure}

%Given a net of $N$ nodes with capacitance $C_1, C_2, ... C_N$, and the output slew $s_o$ of node $o \in O$, the input slew $s_i$ of the wire can be estimated using the formula presented in~\cite{TAU:Contest}.
%\begin{equation}
%\label{itoo}
%s_i = \sqrt{s_o^2 + d_o^2 - 2\beta_o}
%\end{equation}
%where $d_o$ is the Elmore Delay from the input to the output:
%\begin{equation}
%d_o = \Sigma_{k \in N}R_{shared}C_k,
%\end{equation}
%and $\beta$ can be calculated using the same equation, but replacing all $C_k's$ term by $C_kd_k$:
%\begin{equation}
%\beta_o = \Sigma_{k \in N}R_{shared}C_kd_k.
%\end{equation}
%For multi-output nets, the $s_i$ value should be taken as the minimum across the values calculated by given different output $o$ because all the output slew target have to be satisfied. In other words,
%\begin{equation}
%\label{itoo}
%s_i = \min_{o \in P}\{\sqrt{s_o^2 + d_o^2 - 2\beta_o}\}
%\end{equation}

%The algorithm terminates when the stopping criterion is met: 1. the worst slack becomes worse and 2. the absolute sum of negative slacks divided by the number of end points and the average cell area increases. Then, the assignment of the previous iteration, which achieves the best present objective value, is recovered. Such a method avoids incremental timing updates, and is thus more timing efficient.




\section{Parallelization}
\label{sec:parallel}

\subsection{The Operator Formulation}

We leverage parallelization to achieve fast design optimization. In order to do so effectively, we analyze the parallelism available in our optimizer with the operator formulation~\cite{pingali11}, a {\em data-centric} abstraction of algorithms.

The operator formulation starts from identifying the data structures involved in the algorithm, e.g., a graph. {\em Active elements} capture where in the graph the computation needs to be done. An {\em operator} specifies the rules to update the graph, and it will be applied to active elements. Each application of an operator to an active element is called an {\em action}. An action may need to read from or write to a set of nodes and edges around the active element, which is termed the {\em neighborhood} of the action. Active elements become inactive once the actions are finished.

Algorithms can be categorized as {\em data-driven} or {\em topology-driven} based on the pattern of active elements. A data-driven algorithm begins with a set of initially active elements, generates new active elements on the fly, and terminates when there are no more active elements to be processed. Dijkstra's single-source shortest path (SSSP) algorithm is an example. In contrast, a topology-driven algorithm makes sweeps over all nodes/edges until certain convergence criteria is reached. Bellman-Ford algorithm is an example.

Scheduling needs to be considered when there are multiple active elements at the same time. For {\em unordered} algorithms, e.g., chaotic relaxation for SSSP, processing active elements in any order gives the same answer. However, some ordering may be more efficient than the others.

%{\em Ordered} algorithms, on the other hand, require active elements to be processed as if by certain ordering in order to give correct results~\cite{hassaan11}. For instance, discrete event simulation needs to follow time ordering among events.

Parallelism in graph algorithms can be exploited among actions with disjoint neighborhoods and read-only operators.

\subsection{Available Parallelism}
\label{sec:avail_parallelism}

With the operator formulation of algorithms, we are ready to analyze the parallelism available in our design optimizer, composed of static timing analysis, buffer insertion, and gate sizing using slew targeting.

\subsubsection{Static Timing Analysis}
\label{sec:sta_parallel}

Given a synchronous design, STA represents the design as a timing graph $G = (V, E)$, a DAG, where nodes in $V$ are the pins and $(u, v) \in E$ are wires or timing arcs between pins. For an edge $(u, v)$, we say $u$ is $v$'s predecessor and $v$ is $u$'s successor. STA then computes for all pins their arrival times and slew rates in topological order from primary inputs; and computes required times and slacks in reverse topological order from primary outputs and constrained pins, e.g., register inputs.

As the timing graph for a synchronous design is a DAG, the processing order in STA can be enforced by explicitly tracking the number of unresolved dependencies for each pin. A pin can be processed if all its dependencies are resolved, and all the pins whose dependencies are cleared at the same time can be processed in parallel.

Specifically, each pin $v$ is associated with a counter, $dep(v)$, to track the number of unresolved dependencies for $v$ at a moment. For computing arrival times and slew rates, initially $dep(v) = |pred(v)|$, where $pred(v)$ is the set of $v$'s predecessors. When a pin $u$ finishes computing its arrival time and slew rate, $u$ atomically decrements $dep(v)$ for all $v \in succ(u)$, where $succ(u)$ is the set of $u$'s successors. Pin $v$ becomes active when $dep(v) = 0$. Initially only primary inputs are active. Pins can be processed in parallel if their $dep$s are zero at the same time.

For computing required times and slacks, initially $dep(v) = |succ(v)|$. When a pin $w$ finishes computing its required time and slack, $w$ atomically decrements $dep(v)$ for all $v \in pred(w)$. Pin $v$ becomes active when $dep(v) = 0$. Initially only pins with no successors, e.g., primary outputs, are active. Pins whose $dep$s are zero at the same time can be processed in parallel.

As STA needs to track active nodes explicitly, STA is a {\em data-driven} algorithm. All active pins can be processed in any order, so STA is an {\em unordered} algorithm. Note that the ordering in STA only defines when pins should become active but not the processing order of active pins.

\subsubsection{Buffer Insertion}

\paragraph{For fixing maximum load constraints} As explained in Section~\ref{sec:buf_insert}, a buffer is inserted in between a gate output pin, $v$, and its load, $load(v)$, if $load(v) > maxC(v)$, where $maxC(v)$ is the maximum load that can be driven by $v$.
Every gate output can insert such a buffer independently, so this is a {\em topology-driven}, {\em unordered} algorithm.

\paragraph{For fixing hold time violations} Recall from Section~\ref{sec:buf_insert} that we insert buffers to a design to fix hold time violations round by round. In each round, we run STA to identify the most-critical hold-time path, and insert buffers to the wire segment having the largest setup-time slack on the path. Since there is only one active edge in a round, our buffer insertion scheme contains no parallelism. It is a {\em data-driven} algorithm, since the active edge in a round depends on the circuit timing in the round.

\subsubsection{Gate Sizing with Slew Targeting}

Recall from Section~\ref{sec:gate_sizing} that gate sizing with slew targeting~\cite{Held:Gate} is a round-based algorithm. In each round, a full STA is performed first, then all pins set their slew targets, and then all gates select their cells. Finally, the new cell assignment is scored and then kept or reverted.

\paragraph{Setting slew targets} Slew targets can be set for each pin independently in any order, since each pin $p$ can get from STA the required information for computing new slew target in the next round: current slew, $p$'s own slack, and the neighboring gates' pins' slacks. Therefore, setting slew targets is a {\em topology-driven}, {\em unordered} algorithm.

\paragraph{Cell assignment} The parallelism available in assigning cells to gates is similar to that in computing required times and slacks in STA. As mentioned in~\ref{sec:sizing_order}, gates should be sized in reverse topological order on the graph of gate connectivity, with edges feeding into register data inputs ignored. Instead of constructing a gate connectivity graph and then ignore some edges in it, we can enforce the order of sizing gates with the idea that a gate should be processed after all its pins are processed. This reduces the memory space requirement of ParallelClosure by utilizing the timing graph for STA for gate sizing as well.

Specifically, we associate for each gate $g$ a counter, $untouched(g)$, to track the number of untouched pins for $g$. Initially $untouched(g) = |pin(g)|$, where $pin(g)$ is the set of pins belonging to gate $g$. The dependency among pins are tracked as in Section~\ref{sec:sta_parallel} for computing required times and slacks in STA. When a pin $v$ is processed, $v$ also atomically decrements $untouch(gate(v))$, where $gate(v)$ is the gate to which pin $v$ belongs. Gate $g$ becomes active when $untouched(g) = 0$. Initially, no gates are active, and pins with no successors are active. All gates with $untouched = 0$ simultaneously can be assigned to cells in parallel.

Similar to STA, cell assignment is a {\em data-driven}, {\em unordered} algorithm. Nevertheless, its operator is different from the one for computing required times and slacks in STA.

\paragraph{Scoring a cell assignment} As mentioned in Section~\ref{sec:sizing_converge}, a new cell assignment is scored as a linear combination of worst negative slack, average total negative slack, and average cell area. The score can be computed by dividing the gates to threads to compute thread-local results, and then reduce all the thread-local results to the final answer. All gates and constrained pins can be processed in parallel. Therefore, scoring a cell assignment is a {\em topology-driven}, {\em unordered} algorithm.

\paragraph{Keeping/reverting a cell assignment} Each gate can be processed independently, so this is a {\em topology-driven}, {\em unordered} algorithm.

\subsection{Implementation in Galois}

We implement our design optimizer using the Galois framework~\cite{nguyen:2013,Lenharth:2016}, a C++ library for parallel programming based on the operator formulation. The Galois framework (1) provides parallel data structures, and language constructs for highlighting parallelization opportunities; and (2) supports dynamic work generation, load balance, resource management, and transactional execution of operators.

All sub-algorithms in our design optimizer are implemented as described in Section~\ref{sec:avail_parallelism} except for buffer insertion, where everything is done sequentially to maintain the mapping consistency from names to gates/wires. The timing graph is constructed as follows: gates track their pins; pins track their wires and belonging gates, if any; and wires track their member pins. Cell assignment is associated with gates and pins consistently. All timing corners and delay modes are analyzed simultaneously for all algorithms. This is done by storing MCMM data for all pins, and processing all MCMM data when processing a pin/gate.

\section{Limitation}
\label{sec:limit}

ParallelClosure, our design optimizer for this contest, currently has the following limitations.

\begin{itemize}
\item The buffer insertion is purely sequential. This will be addressed by changing the data structures for representing Verilog netlists and SPEF wire models so that concurrent updates to the circuit connectivity can scale and do not corrupt the mapping consistency in between names and gate/wires.
\item It inserts lots of buffers when there is a large number of paths with hold time violations. This makes the quality of results for the optimized designs bad, and also makes ParallelClosure running slow. This will be addressed in the future with techniques related to clock network, such as clock skew scheduling~\cite{Friedman:Clock}.
\item The buffer insertion in ParallelClosure currently neither considers the wire topology for a net nor the optimal buffer insertion for a net. This will be addressed by incorporating tree topology generation for a net, e.g., C-tree as in~\cite{Alpert:Buffered}; and van Ginneken's algorithm for optimal buffer insertion given a tree topology~\cite{Lukas:Buffer}. 
\item The parameters needs to be further tuned. For example, the convergence criteria of gate sizing can be tuned so that the sizing can proceed for more than two rounds. Right now gate sizing in ParallelClosure is more like the recovery mechanism for area and leakage power.
\end{itemize}

\section{Conclusions}
\label{sec:conclusions}

In this contest, we explore existing efficient buffer insertion and gate sizing algorithms in the literature to fix timing violations; and optimize area, leakage power, and clock period for a design without changing its functionality. We prototype a design optimizer, ParallelClosure, by incorporating the idea presented in~\cite{Shenoy:Minimum} to insert buffers for fixing hold time violations; and by generalizing the gate sizing by slew targeting~\cite{Held:Gate} to optimize for area and leakage power while not introducing setup or hold time violations. We analyze the parallelism available in STA and gate sizing by slew targeting, and use the Galois framework to parallelize them in ParallelClosure. We believe that ParallelClosure is an efficient and effective design optimizer for designs with a small number of timing violations.

\bibliographystyle{ieeetr}
\bibliography{cite,iss}
\end{document}



