\documentclass[conference]{IEEEtran}
\ifCLASSINFOpdf
\usepackage[pdftex]{graphicx}
\DeclareGraphicsExtensions{.eps}
\else
\usepackage[dvips]{graphicx}
\usepackage{multirow}
\DeclareGraphicsExtensions{.eps}
\fi
\usepackage{booktabs}
\RequirePackage[pdftex]{hyperref}
\usepackage[autostyle]{csquotes}
\usepackage{hyperref}
\usepackage{url}
\usepackage{tikz}
\usepackage{circuitikz}
\usepackage{caption}
\usetikzlibrary{calc, matrix, shapes, arrows, positioning, chains, decorations, shapes.gates.logic.US}
\newtheorem{algorithm}{Algorithm}
\newtheorem{mydef}{Definition}
\newtheorem{proposition}{Proposition}
\newtheorem{remark}{Remark}
\newtheorem{theorem}{Theorem}
\newtheorem{corollary}{Corollary}
\usepackage{algorithm}
\usepackage{algpseudocode}
\usepackage{amsmath}
\usepackage{amsfonts}
\usepackage{bm}
\usepackage{amssymb}
\usepackage{listings}
\usepackage{subcaption}
\lstset{
basicstyle={\normalsize\ttfamily},
escapeinside={(*}{*)},
%emph={function_up :}, emphstyle={\textbf}
}
\usepackage{ctable}

\begin{document}

\title{2019 TAU Contest: Power Performance Area (PPA) Optimization of Design
  Circuits\thanks{Thanks to various funding agencies. Omitted for blind review}
}

\author{\IEEEauthorblockN{Yi-Shan Lu}
\IEEEauthorblockA{\textit{Electrical and Computer Engineering} \\
\textit{University of Texas at Austin}\\
Austin, TX \\
yishanlu@utexas.edu}
\and
\IEEEauthorblockN{Wenmian Hua}
\IEEEauthorblockA{\textit{Electrical Engineering} \\
\textit{Yale University}\\
New Haven, CT \\
wenmian.hua@yale.edu}
%\and
%\IEEEauthorblockN{3\textsuperscript{rd} Given Name Surname}
%\IEEEauthorblockA{\textit{dept. name of organization (of Aff.)} \\
%\textit{name of organization (of Aff.)}\\
%City, Country \\
%email address}
%\and
%\IEEEauthorblockN{4\textsuperscript{th} Given Name Surname}
%\IEEEauthorblockA{\textit{dept. name of organization (of Aff.)} \\
%\textit{name of organization (of Aff.)}\\
%City, Country \\
%email address}
}
\maketitle

\begin{abstract}
Timing-driven optimization is essential for the success of closure flows. In this contest, we explored existing efficient buffer insertion and gate sizing algorithms to fix timing violations of a design and optimize its leakage power, performance and area without affecting its functionality. By using the infrastructure provided by the Galois system for parallelization, we will provide a fast design optimization flow which can be run in part of a static timing engine.
\end{abstract}
\tikzset{
decision/.style={
  diamond,
  draw,
  text width=4em,
  text badly centered,
  inner sep=0pt
},
flipflop/.style={
  rectangle,
  draw,
  very thick,
  minimum height=5em,
  minimum width=1.3em,
},
block/.style={
  rectangle,
  draw,
  text width=6em,
  text centered,
  very thick,
  rounded corners,
  align=center,
},
cycle/.style={
  ellipse,
  draw,
  dashed,
  minimum width = 1.6cm,
  minimum height = 1.05cm,
  ultra thick,
},
Celement/.style={
  circle,
  draw,
  very thick,
  minimum width = 0.7cm,
},
combinational/.style={
  cloud,
  draw,
  very thick,
  minimum height=4.5em,
  minimum width=1.8cm,
},
module/.style={
  rectangle,
  draw,
  very thick,
  minimum height=5em,
  minimum width=1.2cm,
  inner sep=0pt,
},
cell/.style={
  rectangle,
  draw,
  very thick,
  minimum height=4.5em,
  minimum width=1cm,
  inner sep=0pt,
},
descr/.style={
  fill=white,
  inner sep=2.5pt,
},
CLK/.style={
  isosceles triangle,
  isosceles triangle apex angle=60,
  draw,
  very thick,
  minimum height=0.27cm,
  inner sep=0pt,
},
arrow/.style={
  ->, >=stealth, thick,
},
connector/.style={
  -latex,
  font=\scriptsize,
},
line/.style={
  draw, -latex',
  very thick,
  inner sep=0pt,
},
capa/.style={
  draw=black,
  line,
  minimum width=0.5cm,
  inner sep=0pt,
},
groud/.style={
  draw=black,
  line,
  minimum width=0.25cm,
  inner sep=0pt,
},
}
\section{Introduction}
Timing-driven optimization is imperative for the success of closure flows, and it is essential to find efficient ways to make changes to the design to satisfy timing constraints and optimize power, performance and area. In this contest, given a Verilog netlist, a SPEF file for net delay models, a set of cell libraries (corners) and an SDC file, we are going to generate an optimized design in Verilog and Spef with no violations of hold time and maximum slew transition, and minimized area, leakage power and clock period for all given corners. Note that the optimization should consider multi-corner multi-mode (MCMM) scenario.

The optimization can be done by buffer insertion/deletion and gate sizing. 


According to the paper~\cite{Jiang:Interleaving}, buffer insertion implies an increase of the area, and might be more preferred for highly critical paths. For other mildly critical paths, however, gate sizing will be better choice. Thus, we are going to implement our design optimization flow in this contest as following: First, we run timing analysis for the whole circuit once, and identify high fan-out nets (and highly critical nets) with some reasonable criterions. Then based on the contest rule that no buffers are allowed within an RC-tree, we will perform buffer insertion for those high fan-out nets without an RC-tree only and leave the remaining violations and optimization problems to gate sizing stage for convenience. Finally, we will perform the gate sizing and conclude our design optimization flow.



%Also, according to the literature review, slew target method gives more efficient and accurate gate sizing solutions. Furthermore, our Galois system provides better support for graph-based algorithms. Considering all of these facts, we


\section{Methodologies}

%\subsection{Buffer Insertion}
%%For buffer insertion problem, van Ginneken's algorithm~\cite{Lukas:Buffer} is considered to be a classic in the field. Given a fixed buffer tree and candidate buffer locations, it finds the optimal buffer solutions under Elmore delay models. Several papers extend the algorithm to handle library with multiple buffers, various kinds of constraints and delay models and wire sizing integration~\cite{Lillis:Optimal,Alpert:Wire,Alpert:Buffer,Lillis:Simultaneous,Cong:Buffer,Wong:A}. A significant limitation of van Ginneken's algorithm is that a fixed Steiner topology must be provided in advance. Several papers address the tree construction problems, including C-Tree Algorithm~\cite{Alpert:Buffered}, P-Tree Algorithm~\cite{Lillis:New} and S-Tree Algorithm~\cite{Hrkic:S}.
%
%In this contest, to minimize the area when inserting buffers, we only choose the smallest buffer available in the cell library. Also since we are considering the problem under multiple corners, we will choose the $\star$ worst corner to perform buffer insertion. Notice that this step is not aim to fix all the timing violations, but to just reduce the capacitance load of those high fan-out nets, and any remaining violations and optimization problems will be left to gate sizing section. 
%
%We adopt the ideas presented in C-Tree Algorithm~\cite{Alpert:Buffered}, followed by van Ginneken's algorithm~\cite{Lukas:Buffer}. Since we are only going to insert buffers for high-fan-out nets without an RC-tree (which means ideal wires), we might want to modify the existing algorithm a little bit and we will not construct the specific steiner tree construction as described in~\cite{Alpert:Buffered}. We will start by performing 2-clustering, which separate the sinks with similar characteristics (criticality and capacitance) into two clusters recursively, from top to the bottom, until each cluster has no more than 2 sinks. This step essentially create a binary tree where each branching point is a potential position for a buffer. Then, the van Ginneken's algorithm~\cite{Lukas:Buffer} can be applied to determine the optimal buffer insertion solution for the given topology.
%
%Since the solution quality heavily depends on the initial topology we construct, we need to give a good clustering metric between pairs of points (sinks). Due to the ideal wire-model assumption of these nets for buffer insertion in this contest, our clustering metric is slightly different from that presented in~\cite{Lukas:Buffer}. In our implementation, sinks nodes are clustered based on the slack and the capacitance value. We adopt similar idea presented in the paper and the criticality distance between two sink nodes $s_i$ and $s_j$, $tDist(s_i, s_j) = |crit(s_i) - crit(s_j)|$ and
%\begin{equation}
%crit(s_i) = e^{\alpha(mAS - AS(s_i))/(aAS - mAS)}
%\end{equation}
%where $AS(s_i)$ is the slack of the sink node $s_i$, $mAS$ and $aAS$ are the minimum and the average slack for all the sinks, and $\alpha > 0$ is a user parameter. Since the buffers are inserted in an ideal wire in our problem, we also have $AS(s_i) = RAT(s_i)$, so are $mAS$ and $aAS$. Furthermore, our problem consider multiple corners, and we will just choose the corner which gives the worst slack for each sink so that the algorithm gives the buffer insertion solution under the worst case.
%
%The capacitance distance definition is different. Since we want to separate the two sinks whenever they lead to large total capacitance, a reasonable normalized capacitance distance definition between two sink nodes $s_i$ and $s_j$, $cDist(s_i, s_j)$ will be $|ncap(s_i) + ncap(s_j)|$, where
%\begin{equation}
%ncap(s_i) = \frac{cap(s_i)}{\Sigma_i cap(s_i)},
%\end{equation}
%The final distance $D(s_i, s_j)$ is simply defined as the summation of the two distances:
%\begin{equation}
%D(s_i, s_j) = tDist(s_i, s_j) + cDist(s_i, s_j).
%\end{equation}
%%we are going to use in clustering will be the Euclidean distance defined as:
%%\begin{equation}
%%D(s_i, s_j) = \sqrt{tDist(s_i, s_j)^2 + cDist(s_i, s_j)^2}.
%%\end{equation}

\subsection{Gate Sizing}
%Gate sizing problems are also well studied in the literature. TILOS~\cite{Fishburn:TILOS} applied Greedy method to gate sizing and modeled delay and slew functions as posynomials. The paper~\cite{Sapatnekar:An} provided a geometric programming approach, which is guaranteed to find the global optimal solution, to solve the transistor sizing problem. Lagrangian Relaxation is also widely used in both continuous gate sizing~\cite{Chen:Fast} and discrete gate sizing~\cite{Liu:A}, and it provides scalable methods to large instances. Faster approaches for gate sizing in practice given predefined discrete cell libraries are delay budget approaches~\cite{Chen:iCOACH,Dai:MOSIZ}, which repeatedly distribute delay budgets to cells and can manage interactions between gates~\cite{Held:Gate}. Slew target can also be distributed instead of delay budgets as an alternative, based on the assumption that the delay is a monotonic function of the slew. Most of other discrete gate sizing methods ignore the slew, and compared to these work, the slew targeting method enables a more accurate gate sizing~\cite{Held:Gate}.

Gate sizing problems are also well studied in the literature. In this contest, we are going to implement the slew target method presented in~\cite{Held:Gate} for gate sizing due to its accuracy, timing efficiency and adaptivity to our Galois framework. More specifically, unlike greedy method implemented in TILOS~\cite{Fishburn:TILOS} and geometric programming approach adopted in~\cite{Sapatnekar:An}, slew target method in~\cite{Held:Gate} is more timing efficient by avoiding incremental timing updates. Slew target approach~\cite{Held:Gate} also turns out to give more accurate gate sizing solutions than delay budget approach~\cite{Chen:iCOACH,Dai:MOSIZ}. Furthermore, it is a graph-based algorithm, which well adapts to our Galois system. Since the paper focuses on fixing setup time violations, we generalize the method a little bit such that it also addresses hold time violations. 

Slew target method is based on the assumption that the delay is a monotonic function of the input slew. This assumption generally holds in practical design as can be seen in the cell library file, but there might be small fluctuations on delay/output slew vs input slew where the monotone property fail. Empirically, we do not have to worry about this because output slew and delay are weak functions of input slew, and the small fluctuations will not affect the accuracy of the final solution of our implementation by much.

%It turns out that slew target method gives more efficient and accurate gate sizing solutions. The assumption that the delay is a monotonic function of the slew generally holds, although some small fluctuations might exist, which should not affect the final solution by much. Furthermore, our Galois system provides better support for graph-based algorithms. Thus, in this contest, we are going to adopt the slew target method presented in~\cite{Held:Gate} for gate sizing.

The slew target method works as following: An initial round of timing analysis is performed for the design and the slew targets under setup time constraint (and hold time constraint) for all cell output pins $p$: $slewt_s(p)$ and $slewt_h(p)$ are initialized as the maximum slew rate (and minimum slew rate) $\star$ indicated in the cell library, respectively. According to the slew targets, we assign the sizes to each cell in reverse topological order, because the output capacitance usually has larger impact on the cell timing than the input slew, and it is better to know the exact output capacitance when sizing a cell. Specifically, look at Figure~\ref{fig:slew}. When sizing the cell $c$, the downstream cells are already sized and the output capacitance is already available. However, the input slew $slew(q)$ also depends on the unknown predecessor size. As a result, the paper gives an estimation of slew at $p'$, under setup time constraint (and hold time constraint): $est\_slew_s(p')$ (and $est\_slew_h(p')$), based on the slew provided by timing analysis and its slew target, to decide the size of $c$:
\begin{equation}
est\_slew_s(p') = \theta slewt_s(p') - (1-\theta)slew(p'),
\end{equation}
and
\begin{equation}
est\_slew_h(p') = \theta slewt_h(p') - (1-\theta)slew(p'),
\end{equation}
where $\theta \in [0, 1]$ is an empirical factor which can shift from pure slew target $\theta = 1$ to real timing analysis $\theta = 0$.

After we have estimated slew for $p'$, we can use the slew degradation formula presented in~\cite{TAU:Contest} to calculate $slew_s(q)$ (and $slew_h(q)$). We can then use the table look-up in the cell library to assign the smallest (and largest) cell size that gives the output slew $slew(p)$ such that $slew_s(p) \leq slewt_s(p)$ (and $slew_h(p) >= slewt_h(p)$) for any given timing arc from an input to an output. As we consider the case when there are multiple outputs and multiple corners, we must take the largest (and the smallest) cell size across different outputs and different corners under setup time target (and hold time target) because all the timing outputs and corners must satisfy the slew target.

The gate sizing solution for the setup time target (and the hold time target) gives an lower bound (and a upper bound) for the cell size. If the intersection is non-empty, which means that there are cell sizes that satisfy both slew targets, then we choose the minimum size within the intersection to minimize the area and the power. If the intersection is empty, then we choose the upper bound gate size, which satisfies the hold time slew target, while sacrifices the setup time slew target to the least extent. This is because hold time constraint is more essential and we can easily fix the setup timing constraint by lower the clock frequency.

After gate(cell) sizing assignment, the timing analysis based on the design with new gate(cell) sizes, and refine the slew target for the next round of gate sizing assignment. This iterative process terminates until the stopping criterion is met: 1. the worst negative slack becomes worse and 2. the absolute sum of negative slacks divided by the number of end points and the average cell area increases. Then, the assignment of the previous iteration, which achieves the best present objective value, is recovered.

%Again, since our problem considers multiple corners, after we get the sizing solution for each corner, we must take the maximum sizing for each cell across different corners to make sure that the sizing solution satisfies the slew targets for all the corners.

%Since the paper focuses on fixing setup time violations. We generalize the method by using similar method such that it also address the hold time violation. Instead, $slewt(p)$ are initialized to 0 and the cell assignment is also opposite to that in setup violation removal step: we choose the maximum cell size that gives the output slew $slew(p)$ such that $slew(p) \geq slewt(p)$, and take the minimum sizing for each cell across different corners to satisfy the slew target for hold time constraint for all the corners. 
%
%The gate sizing solution under setup time constraint gives the upper bound for sizing of each cell, and that under hold time constraint gives the lower bound for sizing of each cell. I
%Initially, when the sizing of predecessors are actively changing, both $slewt(p')$ and $slew(p')$ contribute much to the actual slew at $p'$. Later when the sizing of predecessors become stable,


%The slew targets $slewt(p')$ and $slewt(p)$ are assigned for both $p'$ and $p$ and we can then use the formula presented in~\cite{TAU:Contest} to calculate $q$. To assign a size to the cell $c$, notice that at this time, the downstream cells have all been assigned and the output capacitance seen at $p$ is fixed and known. We can then use the table look-up in the cell library to determine the smallest cell size that gives the output slew $slew(p)$ such that $slew(p) \leq slewt(p)$. After gate(cell) sizing assignment, the timing analysis based on the design with new gate(cell) sizes, and refine the slew target for the next round of gate sizing assignment. This iterative process terminates until some stopping criterion is met.
\begin{figure}
\center
\begin{tikzpicture}[align=center, font=\footnotesize, node distance = 2cm]
  \node [module] (1) {}    ;
  \node [module] (2) [below right=-0.9cm and 1.5cm of 1] {$c$}  ;
  \draw [very thick] (1.east) -- ++(1, 0) node[pos=0.0, above right]{$p'$} -- ++(0, 0.4) -- ++(0.5, 0);
  \draw [very thick] (1.east) -- ++(1, 0) -- ++(0, -0.4) -- ++(0.5, 0) node[pos=0.1, below right]{$q$};
  \draw [very thick] ([yshift=-0.4cm]2.west) -- ++(-0.5, 0);
  \draw [very thick] ([yshift=-0.4cm]1.west) -- ++(-0.5, 0);
  \draw [very thick] ([yshift=0.4cm]1.west) -- ++(-0.5, 0);
  \draw [very thick] (2.east) -- ++(0.5, 0) node[pos=0.0, above right]{$p$};
\end{tikzpicture}
\caption{A Cell $c$ and a Predecessor}
\label{fig:slew}
\end{figure}

%Given a net of $N$ nodes with capacitance $C_1, C_2, ... C_N$, and the output slew $s_o$ of node $o \in O$, the input slew $s_i$ of the wire can be estimated using the formula presented in~\cite{TAU:Contest}.
%\begin{equation}
%\label{itoo}
%s_i = \sqrt{s_o^2 + d_o^2 - 2\beta_o}
%\end{equation}
%where $d_o$ is the Elmore Delay from the input to the output:
%\begin{equation}
%d_o = \Sigma_{k \in N}R_{shared}C_k,
%\end{equation}
%and $\beta$ can be calculated using the same equation, but replacing all $C_k's$ term by $C_kd_k$:
%\begin{equation}
%\beta_o = \Sigma_{k \in N}R_{shared}C_kd_k.
%\end{equation}
%For multi-output nets, the $s_i$ value should be taken as the minimum across the values calculated by given different output $o$ because all the output slew target have to be satisfied. In other words, 
%\begin{equation}
%\label{itoo}
%s_i = \min_{o \in P}\{\sqrt{s_o^2 + d_o^2 - 2\beta_o}\}
%\end{equation}

%The algorithm terminates when the stopping criterion is met: 1. the worst slack becomes worse and 2. the absolute sum of negative slacks divided by the number of end points and the average cell area increases. Then, the assignment of the previous iteration, which achieves the best present objective value, is recovered. Such a method avoids incremental timing updates, and is thus more timing efficient.




\section{Parallelization}
\label{sec:parallel}

\subsection{The Operator Formulation}

We leverage parallelization to achieve fast design optimization. In order to do so effectively, we analyze the parallelism available in our optimizer with the operator formulation~\cite{pingali11}, a {\em data-centric} abstraction of algorithms.

The operator formulation starts from identifying the data structures involved in the algorithm, e.g., a graph. {\em Active elements} capture where in the graph the computation needs to be done. An {\em operator} specifies the rules to update the graph, and it will be applied to active elements. Each application of an operator to an active element is called an {\em action}. An action may need to read from or write to a set of nodes and edges around the active element, which is termed the {\em neighborhood} of the action. Active elements become inactive once the actions are finished.

Algorithms can be categorized as {\em data-driven} or {\em topology-driven} based on the pattern of active elements. A data-driven algorithm begins with a set of initially active elements, generates new active elements on the fly, and terminates when there are no more active elements to be processed. Dijkstra's single-source shortest path (SSSP) algorithm is an example. In contrast, a topology-driven algorithm makes sweeps over all nodes/edges until certain convergence criteria is reached. Bellman-Ford algorithm is an example.

Scheduling needs to be considered when there are multiple active elements at the same time. For {\em unordered} algorithms, e.g., chaotic relaxation for SSSP, processing active elements in any order gives the same answer. However, some ordering may be more efficient than the others.

%{\em Ordered} algorithms, on the other hand, require active elements to be processed as if by certain ordering in order to give correct results~\cite{hassaan11}. For instance, discrete event simulation needs to follow time ordering among events.

Parallelism in graph algorithms can be exploited among actions with disjoint neighborhoods and read-only operators.

\subsection{Available Parallelism}
\label{sec:avail_parallelism}

With the operator formulation of algorithms, we are ready to analyze the parallelism available in our design optimizer, composed of static timing analysis, buffer insertion, and gate sizing using slew targeting.

\subsubsection{Static Timing Analysis}
\label{sec:sta_parallel}

Given a synchronous design, STA represents the design as a timing graph $G = (V, E)$, a DAG, where nodes in $V$ are the pins and $(u, v) \in E$ are wires or timing arcs between pins. For an edge $(u, v)$, we say $u$ is $v$'s predecessor and $v$ is $u$'s successor. STA then computes for all pins their arrival times and slew rates in topological order from primary inputs; and computes required times and slacks in reverse topological order from primary outputs and constrained pins, e.g., register inputs.

As the timing graph for a synchronous design is a DAG, the processing order in STA can be enforced by explicitly tracking the number of unresolved dependencies for each pin. A pin can be processed if all its dependencies are resolved, and all the pins whose dependencies are cleared at the same time can be processed in parallel.

Specifically, each pin $v$ is associated with a counter, $dep(v)$, to track the number of unresolved dependencies for $v$ at a moment. For computing arrival times and slew rates, initially $dep(v) = |pred(v)|$, where $pred(v)$ is the set of $v$'s predecessors. When a pin $u$ finishes computing its arrival time and slew rate, $u$ atomically decrements $dep(v)$ for all $v \in succ(u)$, where $succ(u)$ is the set of $u$'s successors. Pin $v$ becomes active when $dep(v) = 0$. Initially only primary inputs are active. Pins can be processed in parallel if their $dep$s are zero at the same time.

For computing required times and slacks, initially $dep(v) = |succ(v)|$. When a pin $w$ finishes computing its required time and slack, $w$ atomically decrements $dep(v)$ for all $v \in pred(w)$. Pin $v$ becomes active when $dep(v) = 0$. Initially only pins with no successors, e.g., primary outputs, are active. Pins whose $dep$s are zero at the same time can be processed in parallel.

As STA needs to track active nodes explicitly, STA is a {\em data-driven} algorithm. All active pins can be processed in any order, so STA is an {\em unordered} algorithm. Note that the ordering in STA only defines when pins should become active but not the processing order of active pins.

\subsubsection{Buffer Insertion}

\paragraph{For fixing maximum load constraints} As explained in Section~\ref{sec:buf_insert}, a buffer is inserted in between a gate output pin, $v$, and its load, $load(v)$, if $load(v) > maxC(v)$, where $maxC(v)$ is the maximum load that can be driven by $v$.
Every gate output can insert such a buffer independently, so this is a {\em topology-driven}, {\em unordered} algorithm.

\paragraph{For fixing hold time violations} Recall from Section~\ref{sec:buf_insert} that we insert buffers to a design to fix hold time violations round by round. In each round, we run STA to identify the most-critical hold-time path, and insert buffers to the wire segment having the largest setup-time slack on the path. Since there is only one active edge in a round, our buffer insertion scheme contains no parallelism. It is a {\em data-driven} algorithm, since the active edge in a round depends on the circuit timing in the round.

\subsubsection{Gate Sizing with Slew Targeting}

Recall from Section~\ref{sec:gate_sizing} that gate sizing with slew targeting~\cite{Held:Gate} is a round-based algorithm. In each round, a full STA is performed first, then all pins set their slew targets, and then all gates select their cells. Finally, the new cell assignment is scored and then kept or reverted.

\paragraph{Setting slew targets} Slew targets can be set for each pin independently in any order, since each pin $p$ can get from STA the required information for computing new slew target in the next round: current slew, $p$'s own slack, and the neighboring gates' pins' slacks. Therefore, setting slew targets is a {\em topology-driven}, {\em unordered} algorithm.

\paragraph{Cell assignment} The parallelism available in assigning cells to gates is similar to that in computing required times and slacks in STA. As mentioned in~\ref{sec:sizing_order}, gates should be sized in reverse topological order on the graph of gate connectivity, with edges feeding into register data inputs ignored. Instead of constructing a gate connectivity graph and then ignore some edges in it, we can enforce the order of sizing gates with the idea that a gate should be processed after all its pins are processed. This reduces the memory space requirement of ParallelClosure by utilizing the timing graph for STA for gate sizing as well.

Specifically, we associate for each gate $g$ a counter, $untouched(g)$, to track the number of untouched pins for $g$. Initially $untouched(g) = |pin(g)|$, where $pin(g)$ is the set of pins belonging to gate $g$. The dependency among pins are tracked as in Section~\ref{sec:sta_parallel} for computing required times and slacks in STA. When a pin $v$ is processed, $v$ also atomically decrements $untouch(gate(v))$, where $gate(v)$ is the gate to which pin $v$ belongs. Gate $g$ becomes active when $untouched(g) = 0$. Initially, no gates are active, and pins with no successors are active. All gates with $untouched = 0$ simultaneously can be assigned to cells in parallel.

Similar to STA, cell assignment is a {\em data-driven}, {\em unordered} algorithm. Nevertheless, its operator is different from the one for computing required times and slacks in STA.

\paragraph{Scoring a cell assignment} As mentioned in Section~\ref{sec:sizing_converge}, a new cell assignment is scored as a linear combination of worst negative slack, average total negative slack, and average cell area. The score can be computed by dividing the gates to threads to compute thread-local results, and then reduce all the thread-local results to the final answer. All gates and constrained pins can be processed in parallel. Therefore, scoring a cell assignment is a {\em topology-driven}, {\em unordered} algorithm.

\paragraph{Keeping/reverting a cell assignment} Each gate can be processed independently, so this is a {\em topology-driven}, {\em unordered} algorithm.

\subsection{Implementation in Galois}

We implement our design optimizer using the Galois framework~\cite{nguyen:2013,Lenharth:2016}, a C++ library for parallel programming based on the operator formulation. The Galois framework (1) provides parallel data structures, and language constructs for highlighting parallelization opportunities; and (2) supports dynamic work generation, load balance, resource management, and transactional execution of operators.

All sub-algorithms in our design optimizer are implemented as described in Section~\ref{sec:avail_parallelism} except for buffer insertion, where everything is done sequentially to maintain the mapping consistency from names to gates/wires. The timing graph is constructed as follows: gates track their pins; pins track their wires and belonging gates, if any; and wires track their member pins. Cell assignment is associated with gates and pins consistently. All timing corners and delay modes are analyzed simultaneously for all algorithms. This is done by storing MCMM data for all pins, and processing all MCMM data when processing a pin/gate.

\section{Conclusions}
\label{sec:conclusions}

In this contest, we explore existing efficient buffer insertion and gate sizing algorithms in the literature to fix timing violations; and optimize area, leakage power, and clock period for a design without changing its functionality. We prototype a design optimizer, ParallelClosure, by incorporating the idea presented in~\cite{Shenoy:Minimum} to insert buffers for fixing hold time violations; and by generalizing the gate sizing by slew targeting~\cite{Held:Gate} to optimize for area and leakage power while not introducing setup or hold time violations. We analyze the parallelism available in STA and gate sizing by slew targeting, and use the Galois framework to parallelize them in ParallelClosure. We believe that ParallelClosure is an efficient and effective design optimizer for designs with a small number of timing violations.

\bibliographystyle{ieeetr}
\bibliography{cite}
\end{document}



