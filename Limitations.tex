\section{Limitation}
\label{sec:limit}

ParallelClosure, our design optimizer for this contest, currently has the following limitations.

\begin{itemize}
\item The buffer insertion is purely sequential. This will be addressed by changing the data structures for representing Verilog netlists and SPEF wire models so that concurrent updates to the circuit connectivity can scale and do not corrupt the mapping consistency in between names and gate/wires.
\item It inserts lots of buffers when there is a large number of paths with hold time violations. This makes the quality of results for the optimized designs bad, and also makes ParallelClosure running slow. This will be addressed in the future with techniques related to clock network, such as clock skew scheduling~\cite{Friedman:Clock}.
\item The buffer insertion in ParallelClosure currently neither considers the wire topology for a net nor the optimal buffer insertion for a net. This will be addressed by incorporating tree topology generation for a net, e.g., C-tree as in~\cite{Alpert:Buffered}; and van Ginneken's algorithm for optimal buffer insertion given a tree topology~\cite{Lukas:Buffer}. 
\item The parameters needs to be further tuned. For example, the convergence criteria of gate sizing can be tuned so that the sizing can proceed for more than two rounds. Right now gate sizing in ParallelClosure is more like the recovery mechanism for area and leakage power.
\end{itemize}
