\section{Buffer Insertion}
\label{sec:buf_insert}
We use buffer insertion to first fix maximum capacitance violations, and then fix hold time violations.

To fix maximum capacitance violations, we run static timing analysis (STA) for the circuit and identify pins that are driving loads larger than their maximum capacitance limits. Then for each such pin, we insert a buffer in between the pin and the loads. The inserted buffer should be of the minimum size that can drive the load legally. If no such buffer exists, the buffer of the largest size is inserted to reduce the magnitude of violating the maximum capacitance limits.

To fix hold time violations, we adopt the technique presented in~\cite{Shenoy:Minimum}. We first perform STA and sort all the path endpoints by their worst hold time slacks across multiple corners. We then find the endpoint with worst negative hold time slack $s_h$, its corresponding path, and the associated corner. For all the edges on the path, we pick the edge $(i, j)$ that has the largest setup time slack ($s_s$) for the corresponding corner, and then insert buffers in between pins $i$ and $j$. To reduce the number of buffers inserted, we always insert the smallest buffer in the cell library. Since we are given cell libraries with non-linear delay model (NLDM), we approximate the delay of a buffer, $d_{buf}$, with the slew at $i$ and the pin capacitance at $j$. The number of buffers to be inserted is $\max\{1, \lceil\frac{\min\{|s_s|, |s_h|\}}{d_{buf}}\rceil\}$, which minimizes the impact to setup times. We repeat this process until all the hold time slacks are above a predefined threshold. In this contest, this threshold falls in the interval of $[30ps, 80ps]$ depending on the original worst hold time slack of the given design.



%In a circuit, for each pin $p$, $a_p$ denotes its early arrival time, $A_p$ denotes its late arrival time, $r_p$ denotes its early required time and $R_p$ denotes its late required time. For a legal circuit without timing violations, $r_p \leq a_p \leq A_p \leq R_p$. According to the paper~\cite{Shenoy:Minimum}, in any circuit, hold time violations can be fixed by buffer insertions if and only if $A_{i} - a_{i} \leq R_{o} - r_{o}$ where $i$ is a primary input and $o$ is a primary output or a register data input.


%In this contest, we adopt the technique presented in~\cite{Shenoy:Minimum} to insert buffers to remove hold time violations. In a circuit, for each pin $p$, $a_p$ denotes its early arrival time, $A_p$ denotes its late arrival time, $r_p$ denotes its early required time and $R_p$ denotes its late required time. For a legal circuit without timing violations, $r_p \leq a_p \leq A_p \leq R_p$. According to the paper~\cite{Shenoy:Minimum}, in any circuit, hold time violations can be fixed by buffer insertions if and only if $A_{i} - a_{i} \leq R_{o} - r_{o}$ where $i$ is a primary input and $o$ is a primary output or a register data input.

%\begin{theorem}
%In any circuit, let $i_1$ be a primary input and $i_n$ be a primary output such that there is a path $p$ from $i_1$ to $i_n$. The padding problem has a feasible solution if and only if for any such $p$, $A_{i1} - a_{i1} \leq R_{in} - r_{in}$.
%\end{theorem}

%According to the contest rule, buffers can be inserted an ideal wire, before the root of an RC-tree, or after the leaf of the RC-tree, but not within an RC-tree. In our implementation, we begin by an initial run of timing analysis for the circuit and identify pins that are driving loads larger than their maximum capacitance limit. Then for each such pin, we insert a buffer in between the pin and the loads. The inserted buffer should be of the minimum size that can drive the load legally.



% Since we can get the slew value of the edge from the timing analysis, and the capacitance value that the buffer is driving is also known, the two values are good approximations for the input slew and the output capacitance of the buffer, and we can use them to find the approximated delay value $d$ of the buffer. 


%\subsection{Buffer Insertion}
%%For buffer insertion problem, van Ginneken's algorithm~\cite{Lukas:Buffer} is considered to be a classic in the field. Given a fixed buffer tree and candidate buffer locations, it finds the optimal buffer solutions under Elmore delay models. Several papers extend the algorithm to handle library with multiple buffers, various kinds of constraints and delay models and wire sizing integration~\cite{Lillis:Optimal,Alpert:Wire,Alpert:Buffer,Lillis:Simultaneous,Cong:Buffer,Wong:A}. A significant limitation of van Ginneken's algorithm is that a fixed Steiner topology must be provided in advance. Several papers address the tree construction problems, including C-Tree Algorithm~\cite{Alpert:Buffered}, P-Tree Algorithm~\cite{Lillis:New} and S-Tree Algorithm~\cite{Hrkic:S}.
%
%In this contest, to minimize the area when inserting buffers, we only choose the smallest buffer available in the cell library. Also since we are considering the problem under multiple corners, we will choose the $\star$ worst corner to perform buffer insertion. Notice that this step is not aim to fix all the timing violations, but to just reduce the capacitance load of those high fan-out nets, and any remaining violations and optimization problems will be left to gate sizing section.
%
%We adopt the ideas presented in C-Tree Algorithm~\cite{Alpert:Buffered}, followed by van Ginneken's algorithm~\cite{Lukas:Buffer}. Since we are only going to insert buffers for high-fan-out nets without an RC-tree (which means ideal wires), we might want to modify the existing algorithm a little bit and we will not construct the specific steiner tree construction as described in~\cite{Alpert:Buffered}. We will start by performing 2-clustering, which separate the sinks with similar characteristics (criticality and capacitance) into two clusters recursively, from top to the bottom, until each cluster has no more than 2 sinks. This step essentially create a binary tree where each branching point is a potential position for a buffer. Then, the van Ginneken's algorithm~\cite{Lukas:Buffer} can be applied to determine the optimal buffer insertion solution for the given topology.
%
%Since the solution quality heavily depends on the initial topology we construct, we need to give a good clustering metric between pairs of points (sinks). Due to the ideal wire-model assumption of these nets for buffer insertion in this contest, our clustering metric is slightly different from that presented in~\cite{Lukas:Buffer}. In our implementation, sinks nodes are clustered based on the slack and the capacitance value. We adopt similar idea presented in the paper and the criticality distance between two sink nodes $s_i$ and $s_j$, $tDist(s_i, s_j) = |crit(s_i) - crit(s_j)|$ and
%\begin{equation}
%crit(s_i) = e^{\alpha(mAS - AS(s_i))/(aAS - mAS)}
%\end{equation}
%where $AS(s_i)$ is the slack of the sink node $s_i$, $mAS$ and $aAS$ are the minimum and the average slack for all the sinks, and $\alpha > 0$ is a user parameter. Since the buffers are inserted in an ideal wire in our problem, we also have $AS(s_i) = RAT(s_i)$, so are $mAS$ and $aAS$. Furthermore, our problem consider multiple corners, and we will just choose the corner which gives the worst slack for each sink so that the algorithm gives the buffer insertion solution under the worst case.
%
%The capacitance distance definition is different. Since we want to separate the two sinks whenever they lead to large total capacitance, a reasonable normalized capacitance distance definition between two sink nodes $s_i$ and $s_j$, $cDist(s_i, s_j)$ will be $|ncap(s_i) + ncap(s_j)|$, where
%\begin{equation}
%ncap(s_i) = \frac{cap(s_i)}{\Sigma_i cap(s_i)},
%\end{equation}
%The final distance $D(s_i, s_j)$ is simply defined as the summation of the two distances:
%\begin{equation}
%D(s_i, s_j) = tDist(s_i, s_j) + cDist(s_i, s_j).
%\end{equation}
%%we are going to use in clustering will be the Euclidean distance defined as:
%%\begin{equation}
%%D(s_i, s_j) = \sqrt{tDist(s_i, s_j)^2 + cDist(s_i, s_j)^2}.
%%\end{equation}




